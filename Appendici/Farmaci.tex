\chapter{I farmaci}

Al fine di garantire un'ottima riuscita della procedura ed evitare nel bambino il ricordo di un'esperienza spiacevole e talora traumatica, ad oggi si può ricorrere sia ad interventi farmacologici di analgosedazione che ad interventi non farmacologici. Quest'ultimi possono essere utilizzati come metodo complementare o esclusivo, con innumerevoli benefici: dal ridurre l'agitazione preprocedurale e permettere una miglior transizione alla fase di sedazione, al contenere le dosi di farmaco utilizzate nel corso della procedura finanche, in alcuni casi, al prevenire del tutto il ricorso alla sedazione \cite{Uptodate}. Si tratta di approcci cognitivi e comportamentali che racchiudono tecniche di distrazione, rilassamento, desensibilizzazione e rinforzo positivo; risulta inoltre fondamentale, al fine di abbassare i livelli di ansia, instaurare una relazione di fiducia con i genitori e il bambino e definire un'adeguata strategia comunicativa, che adotti un linguaggio adeguato all'età o eventualmente introduca il gioco come strumento per prendere confidenza con le attrezzature mediche e le varie fasi della procedura. Tuttavia la sedazione farmacologica rimane la principale risorsa per agevolare le procedure più invasive. 

La scelta dell'agente farmacologico incide fortemente sulla qualità del risveglio post procedurale e va basata su diversi fattori. Premesso che il farmaco o la combinazione di farmaci selezionati deve indurre sedazione e analgesia adeguate a consentire il completamento della procedura con successo, mantenendo i riflessi protettivi delle vie aeree e l'autonomia cardiorespiratoria \cite{Krauss2006}, nel processo di decisione è importante valutare anche il livello di profondità della sedazione desiderato, conseguenza a sua volta del grado di analgesia richiesto dalla procedura. Infatti, talvolta è sufficiente solamente l'ansiolisi, altre volte è necessaria una più estesa analgesia, in altre occasioni invece lo scopo è quello di mantenere il paziente immobile, senza quindi bisogno di trattare il dolore. Altri elementi che guidano nella scelta del farmaco sono la familiarità nell'utilizzo da parte del sedatore e le caratteristiche individuali del paziente (età, comorbidità, allergie, grado di collaborazione, {\color{orange} ore di digiuno (?)}), che possono controindicare alcuni agenti farmacologici \cite{Uptodate}.
Infine, sono rilevanti alcune caratteristiche farmacocinetiche: sono infatti preferenziali farmaci con molteplici vie di somministrazione, induzione rapida e breve emivita, tale da concedere un celere recupero con assenti o minimi effetti collaterali.

Di seguito verranno dunque analizzate e confrontate le proprietà, i dosaggi, le indicazioni e gli effetti collaterali dei farmaci più comunemente utilizzati al di fuori della sala operatoria. 

\section{Propofol}

Il propofol è un sedativo ipnotico, non analgesico, il cui meccanismo d'azione a livello del sistema nervoso centrale non è del tutto noto, anche se ne è stata studiata l'interazione con il recettore A dell'acido $\gamma$-amminobutirrico (GABA) \cite{Propofol2015}.

\subsection*{Farmacodinamica}

Il legame tra le molecole di propofol e il recettore ionotropo GABA-A è responsabile degli effetti centrali del farmaco. Questo recettore è infatti un canale per il cloro presente a livello postsinaptico di molti neuroni. Una volta riconosciuto il ligando, si verifica un aumento del flusso degli ioni cloro attraverso la membrana, determinando iperpolarizzazione del neurone e {\color{orange}{refrattarietà}} agli stimoli esterni \cite{Propofol2015}.

\subsection*{Farmacocinetica}

Uno dei motivi per cui il propofol è ampiamente utilizzato riguarda proprio le sue vantaggiose caratteristiche farmacocinetiche: ha infatti un esordio d'azione estremamente rapido (30-45 secondi) ed un altrettanto rapido risveglio (5-15 minuti dopo singolo bolo, più lento in caso di infusione continua o boli multipli) \cite{Simeupsedazione, Uptodatepharmacology}. Si tratta di una molecola liposolubile, che supera la barriera ematoencefalica, macroscopicamente il propofol assume un aspetto lattescente e può essere somministrato solo per via endovenosa. Nel momento dell'iniezione può causare bruciore locale, che può essere prevenuto utilizzando una vena di calibro maggiore o diluendo a metà il propofol con soluzione fisiologica (5 mg/mL) \cite{Simeupsedazione}. 

\subsection*{Posologia}

Viene somministrato con un dosaggio di 1-2 mg/kg come bolo iniziale d'induzione, seguito da successivi boli multipli da 0.5 a 1 mg/kg ogni 2-3 minuti. Per le procedure più lunghe può anche essere utilizzato in infusione continua.
La dose massima è di 10 mg/kg totali per procedura \cite{Simeupsedazione}.

\subsection*{L'associazione con la ketamina}

Se la procedura è dolorosa il propofol può essere dato in combinazione con la ketamina, che oltre ad avere un'importante azione analgesica controbilancia gli effetti ipotensivo e bradicardizzante del propofol. Esistono in commercio delle formulazioni chiamate \emph{ketofol} con diversi rapporti di ketamina e propofol, tra i vantaggi risalta anche una minor incidenza di vomito e agitazione al risveglio, poiché l'effetto combinato permette di risparmiare ketamina \cite{Simeupsedazione}.

\subsection*{Effetti collaterali}

I principali effetti collaterali del propofol sono rappresentati dalla depressione respiratoria e dalla conseguente apnea, che dipendono dal dosaggio, dalla velocità di somministrazione e dalla risposta soggettiva del paziente. Inoltre, il propofol può determinare ipotensione e più raramente bradicardia \cite{propofolsafety2010}.
Un altro temibile quanto raro effetto avverso è costituito dalla \emph{sindrome da infusione di propofol}, potenzialmente fatale e descritta sia negli adulti che nei bambini \cite{Propofolinfusionsyndrome2019}. Si tratta di un insieme di segni e sintomi che si verifica in pazienti critici, che ricevono propofol in infusione a dosaggi elevati (>5 mg/kg/h) o per un periodo prolungato (>48h) ed è caratterizzata dalla presenza di una o più delle seguenti alterazioni, non spiegabili altrimenti: acidosi metabolica, rabdomiolisi, variazioni elettrocardiografiche associate o meno ad AKI, iperkaliemia, dislipidemia, scompenso cardiaco, febbre, elevazione degli enzimi epatici o del lattato. 

\section{Ketamina}

\lipsum[2]

\section{Midazolam}

\lipsum[3]

\section{Dexmedetomidina}

\lipsum[4]

\section{Protossido di azoto}

\lipsum[5]
 