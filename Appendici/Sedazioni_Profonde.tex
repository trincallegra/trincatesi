\chapter{La sedazione procedurale}

La sedazione procedurale è una pratica sempre più diffusa ed utilizzata in una moltitudine di setting in tutto il mondo, anche da professionisti senza una formazione anestesiologica specifica. In ambito pediatrico, considerata l'importanza di offrire analgesia ed ansiolisi, tenendo conto anche delle tappe dello sviluppo psicofisico del bambino e al fine di prevenire la memoria di un vissuto negativo, viene quotidianamente eseguita per un vasto numero di procedure, sia in elezione che in urgenza. 

\section{Definizione}

La sedazione è uno stato di depressione della coscienza, con contestuale riduzione di vigilanza e consapevolezza, indotto dall'utilizzo di farmaci sedativi, ipnotici e dissociativi\footnote{I farmaci più comunemente utilizzati sono riportati nell'appendice B.} con o senza proprietà analgesiche. Questa condizione progredisce, senza una divisione arbitraria, lungo un \emph{continuum}: da un grado di sedazione minimo (o ansiolisi) ad un livello ben più profondo di anestesia generale, che richiede il supporto anestesiologico, attraversando le tappe della sedazione moderata e profonda \cite{Krauss2006}. \`E compito del sedatore conoscere le proprietà dei farmaci utilizzati e titolarli adeguatamente al fine di ottenere il livello di sedazione anticipatamente designato.  
\\Si definisce \emph{procedurale} quando viene attuata al di fuori del teatro operatorio per lo svolgimento di procedure diagnostiche e/o terapeutiche di breve durata, con il raggiungimento di un grado massimo di sedazione profonda e quindi con il mantenimento della funzionalità cardiorespiratoria e dei riflessi protettivi delle vie aeree. 

Nonostante non esista un preciso confine tra un livello di sedazione ed il successivo, i principali stati di depressione della coscienza ottenibili possono essere descritti come segue \cite{Statpearls}:

\begin{description}
\item[Analgesia] Trattamento del dolore senza alterazione intenzionale dello stato mentale.
\item[Sedazione minima] Anche detta ansiolisi; il paziente è sveglio e risponde normalmente allo stimolo verbale. Le funzioni cognitive e di coordinazione potrebbero essere minimamente alterate, mentre la funzione cardiorespiratoria non è influenzata.
\item[Sedazione moderata] Il paziente ha un livello di coscienza depresso ma risponde espressamente allo stimolo verbale, eventualmente accompagnato da un leggero stimolo tattile. Le vie aeree vengono mantenute pervie senza necessità di intervento e la funzionalità cardiorespiratoria è inalterata. 
\item[Sedazione profonda] Il paziente risulta difficilmente risvegliabile ma potrebbe rispondere a ripetuti stimoli verbali o dolorosi. Inoltre, potrebbe necessitare di supporto per mantenere le vie aeree pervie mentre la funzionalità cardiorespiratoria è solitamente inalterata. 
\item[Anestesia generale] Il paziente non è risvegliabile e necessita di supporto per mantenere le vie aeree pervie ed una ventilazione adeguata, infatti si accompagna solitamente a depressione respiratoria e cardiovascolare. 
\item[Sedazione dissociativa] La sedazione con ketamina rappresenta un'eccezione al continuum tra i diversi livelli di sedazione su cui si muovono gli altri agenti farmacologici. Infatti, i suoi effetti terapeutici sono correlati a specifici dosaggi, rappresentati nella tabella~\ref{tab:1}. Dal punto di vista sedativo, porta il paziente in uno stato catalettico, simile alla trance, in cui il paziente è insensibile agli eventi esterni e sperimenta profonda analgesia ed amnesia, tuttavia rimane sveglio e mantiene intatti la respirazione, i riflessi protettivi e la stabilità cardiopolmonare.

\end{description}

\section{Gli scopi}

I principali obiettivi con cui si effettuano le sedazioni procedurali sono \cite{Uptodatesed}: 

\begin{itemize}
    \item Garantire il benessere e l'incolumità del paziente durante tutte le fasi della procedura
    \item Trattare l'ansia, indurre amnesia ed evitare un possibile trauma psicologico associato ad un'esperienza spiacevole e di difficile comprensione per il paziente pediatrico
    \item Ridurre al minimo la percezione del dolore ed evitare un'eventuale risposta vagale alla procedura dolorosa
    \item Controllare il movimento e permettere la riuscita della procedura in modo sicuro ed efficace

\end{itemize}

\section{L'esecuzione}

%% Final test about coautorship, when pushing from overleaf...