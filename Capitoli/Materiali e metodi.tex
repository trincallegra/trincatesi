\chapter{Materiali e Metodi}
%\`E

Questo lavoro di tesi è basato su uno studio osservazionale prospettico condotto presso l'ospedale universitario di terzo livello, istituto materno infantile --- IRCSS Burlo Garofolo di Trieste. In tale analisi si esamina il livello di soddisfazione percepito dal personale infermieristico in merito alla qualità della sedazione pediatrica al di fuori della sala operatoria, mettendo a confronto quattro agenti sedativi ed analgesici tra i più ampiamente utilizzati: propofol, midazolam, ketamina e dexmedetomidina. 
\\Dal momento che in letteratura non esiste uno strumento validato per la valutazione della percezione infermieristica, è stato formulato da un gruppo di pediatri, anestesisti pediatrici ed infermieri pediatrici un questionario appositamente designato agli scopi di questo studio. Dopo essere stato testato su un campione ridotto di 30 soggetti ed essere stato rifinito e corretto sulla base delle raccomandazioni fornite da questi primi partecipanti, è stato infine sottoposto a 51 infermieri esperti nel campo della sedazione peri-procedurale.

\section{Scelta dell'infermiere come indicatore di qualità}

La scelta di valutare la percezione del personale infermieristico deriva dalla necessità di ampliare ed oggettivare i risultati emersi nello studio riguardante la soddisfazione parentale, ampiamente descritto poco sopra \cite{Cortellazzo2022}. Infatti, mentre il caregiver viene facilmente influenzato da diversi fattori associati alla totalità del vissuto ospedaliero, dal servizio di parcheggio, alla simpatia del personale, finanche alla comparsa di complicanze correlate specificamente alla procedura; l'infermiere assiste mensilmente innumerevoli pazienti durante tutte le fasi della procedura: dall'ammissione, al posizionamento dell'accesso venoso, al monitoraggio dei parametri vitali e alla somministrazione dei farmaci prima e durante la sedazione, fino poi al completo risveglio ed alla dimissione. Risulta quindi il candidato più adatto a giudicare in maniera oggettiva gli elementi associati alla sedazione ed al risveglio post procedurale. Inoltre, mentre il genitore esprime il proprio parere basandosi su un'unica o su limitate esperienze di analgosedazione, l'infermiere partecipa a molteplici sedazioni procedurali ogni mese ed è quindi in grado di confrontare diversi regimi farmacologici ed esprimere un giudizio qualitativo basato sull'efficacia della sedazione, sulla facilità e sulla sicurezza della via di somministrazione, oltre che sull'incidenza e la severità degli effetti avversi associati.
\\In conclusione, l'opinione dell'infermiere riveste un ruolo chiave non solo al fine di ottenere delle indicazioni utili per la scelta farmacologica ma anche come occasione per integrare le conoscenze mediche alle osservazioni di una figura di riferimento fondamentale per il paziente e la famiglia, quale quella dell'infermiere. La collaborazione tra la professionalità medica ed infermieristica rappresenta il presupposto migliore per garantire alla comunità una sempre maggiore attenzione alla qualità di cura. 



%in maniera risulta un candidato più adatto a giudicare in modo oggettivo  Inoltre, 



%Dalle conclusioni dello studio sopra descritto sulla soddisfazione parentale in merito alla sedazione procedurale pediatrica \cite{Cortellazzo2022}, è emersa la necessità di estendere l'analisi ad una figura professionale che possa confrontare fornire un giudizio oggettivo. Infatti, il caregiver può essere influenzato nella valutazione finale dall'esperienza medico ospedaliera vissuta nella sua interezza, che può includere alcune variabili non strettamente associate alla sedazione in sé, quali ad esempio la facilità di parcheggiare, la simpatia del personale, il coinvolgimento nel processo decisionale ed eventuali problematiche specificamente collegate alla procedura. Inoltre, esprime la propria opinione sulla base, nella maggior parte dei casi, di un'unica o di limitate esperienze di analgosedazione e possiede, quindi, scarse o nulle conoscenze rispetto ai regimi farmacologici alternativi. 
%La scelta dell'infermiere come indicatore di qualità della sedazione procedurale ha il compito sia di rispondere all'esigenza di ottenere dei risultati basati su un giudizio più oggettivo, sia di rappresentare un'opportunità di migliorare lo standard di cure e la qualità del risveglio post sedazione. Gli infermieri arruolati in questo lavoro di tesi sono tutti professionisti esperti nel campo della sedazione peri-procedurale, prendono parte mensilmente ad un numero variabile di procedure attuate con regimi farmacologici differenti, assistendo il paziente nelle fasi che precedono e seguono la procedura, fino al completo risveglio ed alla dimissione; inoltre posizionano i cateteri venosi periferici e cooperano con il medico durante la sedazione somministrando i farmaci e monitorando i parametri vitali del bambino. Concludendo, l'infermiere risulta il candidato più adatto a comparare i diversi regimi sedativi e può rappresentare la chiave di volta nel fornire delle indicazioni imparziali, che possano essere utili al sedatore per la scelta farmacologica. Inoltre, questa analisi pone le basi per un potenziare la collaborazione tra medico ed infermiere e garantire al paziente ed alla famiglia una sempre maggiore attenzione alla qualità di cura.

%al fine di migliorare lo standard di cure e la qualità del risveglio post sedazione
%A tale proposito l'infermiere potrebbe essere il candidato più adatto,
%circa la scelta del genitore come indicatore di qualità del servizio sanitario offerto in merito alla sedazione procedurale,
%, invece, potrebbe rispondere all'esigenza di oggettivare la percezione 
%che, comprendendo il giudizio infermieristico, permette di approfondire il rapporto medico-infermiere,

%l'analisi della sua percezione
\section{Questionario}


\section{Analisi Statistica}

%\lipsum[2]