\chapter{Materiali e Metodi}

Questo lavoro di tesi è basato su uno studio osservazionale prospettico condotto presso l'ospedale universitario di terzo livello, istituto materno infantile --- IRCSS Burlo Garofolo di Trieste. In tale analisi si esamina il grado di soddisfazione del personale infermieristico in merito alla qualità della sedazione pediatrica al di fuori della sala operatoria, mettendo a confronto quattro agenti sedativi ed analgesici tra i più ampiamente utilizzati: propofol, midazolam, ketamina e dexmedetomidina. 

\section{La scelta della figura infermieristica}
Dalle conclusioni dello studio sopra descritto riguardante la soddisfazione parentale circa la sedazione procedurale pediatrica \cite{Cortellazzo2022}, è emersa la necessità di estendere l'analisi anche ad una figura professionale che possa fornire un giudizio più oggettivo e competente. Infatti, il caregiver può essere influenzato nella sua valutazione finale dall'esperienza medico ospedaliera nella sua interezza, che può includere alcune variabili non strettamente associate alla sedazione in sé, quali ad esempio la facilità di parcheggiare, la simpatia del personale, la partecipazione nel processo decisionale ed eventuali problematiche specificatamente collegate alla procedura. Inoltre, esprime la propria opinione sulla base, nella maggior parte dei casi, di un'unica o di limitate esperienze di analgosedazione e possiede, quindi, scarse o nulle conoscenze rispetto ai regimi farmacologici alternativi. 
La scelta dell'infermiere come indicatore di qualità della sedazione procedurale ha il compito sia di rispondere all'esigenza di ottenere dei risultati basati su un giudizio più oggettivo, sia di rappresentare un'opportunità di migliorare lo standard di cure e la qualità del risveglio post sedazione. Gli infermieri arruolati in questo lavoro di tesi sono tutti professionisti esperti nel campo della sedazione peri-procedurale, prendono parte mensilmente ad un numero variabile di procedure attuate con regimi farmacologici differenti, assistendo il paziente nelle fasi che precedono e seguono la procedura, fino al completo risveglio ed alla dimissione; inoltre posizionano il catetere venoso periferico e cooperano con il medico durante la sedazione somministrando i farmaci e monitorando i parametri vitali del bambino. Concludendo, l'analisi della percezione infermieristica può rappresentare la chiave di volta nel fornire delle indicazioni imparziali, che possano essere utili al sedatore per la scelta farmacologica, oltre a potenziare la collaborazione tra medico ed infermiere e garantire al paziente ed alla famiglia una sempre maggiore attenzione alla qualità di cura.

%al fine di migliorare lo standard di cure e la qualità del risveglio post sedazione
%A tale proposito l'infermiere potrebbe essere il candidato più adatto,
%circa la scelta del genitore come indicatore di qualità del servizio sanitario offerto in merito alla sedazione procedurale,
%, invece, potrebbe rispondere all'esigenza di oggettivare la percezione 
%che, comprendendo il giudizio infermieristico, permette di approfondire il rapporto medico-infermiere,
\section{Analisi Statistica}

%\lipsum[2]