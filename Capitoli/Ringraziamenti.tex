\setcounter{secnumdepth}{-1}
\chapter{Ringraziamenti}
%\addtocentrydefault{chapter}{}{Ringraziamenti}
%\addcontentsline{toc}{chapter}{Ringraziamenti}

Questo spazio lo dedico alle persone che, con il loro supporto, mi hanno aiutato in questo lungo quanto arricchente percorso universitario. 

\bigskip \noindent
Ringrazio innanzitutto il mio relatore, prof. Barbi, per l'impareggiabile disponibilità e per avermi trasmesso nozioni che non si possono apprendere dai libri.

\bigskip \noindent
Un ringraziamento speciale alla mia correlatrice, dott.ssa Martina D'Agostin, che ha saputo guidarmi con pazienza ed entusiasmo in ogni step della realizzazione di quest'elaborato. %e lasciando spazio alle mie idee. 

\bigskip \noindent
Al mio ragazzo Gabriele: grazie per esserci sempre stato e per aver creduto in me incondizionatamente, anche nei momenti in cui vedevo questo obiettivo gigante e irraggiungibile. Grazie per avermi insegnato a trasformare la paura in coraggio e per avermi spinta ad essere più sicura, consapevole e fiduciosa in me stessa: continuerò comunque a chiederti di leggermi le mail prima di inviarle. Ti ringrazio, inoltre, per aver condiviso con me le tue nozioni di programmazione ed avermi pazientemente aiutato con i contenuti grafici e l'analisi statistica. Senza di te e la tua inarrestabile pignolaggine questo lavoro di tesi non sarebbe altrettanto ben fatto.

%Grazie per avermi insegnato ad avere coraggio, nonostante la paura. Grazie per essere ed essere stato in questi anni il mio compagno di avventure  e il mio migliore amico, semplicemente la mia persona e grazie per avermi dato l'opportunità di arricchire il mio bagaglio di conoscenze, idee, opinioni, grazie alle nostre discussioni e ai nostri confronti sui temi più svariati. Grazie perché se sono orgogliosa della persona che sono oggi è anche merito tuo.  Infine, 

\bigskip \noindent
Ringrazio i miei genitori per avermi sempre sostenuto, supportato e per avermi permesso di portare a termine gli studi universitari, aiutandomi ad affrontare i momenti più difficili. 

\bigskip \noindent
Ringrazio mia sorella Viola per avermi mostrato che dalla costanza e dall'impegno nascono i migliori risultati ma soprattutto per i post-it motivazionali. Non li dimenticherò. 

\bigskip \noindent
Grazie alla mia nonnina Ester, che ha pazientemente atteso questo traguardo e che è quasi più emozionata di me nel vedermi, finalmente, conquistarlo.

\bigskip \noindent
Un grazie sincero ai miei amici Giulia N., Giulia O. e Andrea, che sono sempre stati pronti ad ascoltarmi ed aiutarmi attivamente durante tutte le difficoltà incontrate in questi ultimi mesi. 

\bigskip \noindent
Grazie alle mie coinquiline Fra e Sere che hanno condiviso con me le gioie e i topi di questo percorso, vi voglio bene e vi avrò sempre nel cuore. 

\bigskip \noindent
Grazie ad Ariettis per aver reso spumeggiante e vivernosa quest'esperienza e grazie per tutte le volte che mi hai spinta a buttarmi: senza i tuoi incitamenti ora avrei fatto la metà degli esami.

\bigskip \noindent
Grazie a Jimmy per essere stato la mia colonna sonora del liceo ed essere rimasto il mio compagno di banco, anche se a 170km di distanza. Grazie per aver ascoltato ogni mio singolo audio di 8 minuti e grazie per essere stato il primo ad avermi fatta sentire un "medico": spero che la qualità miei consigli medichesi migliorerà sempre di più. 

\bigskip\noindent
Infine, grazie a tutti i miei amici vicini e lontani, ai miei compagni di università e a tutti quelli che hanno incrociato la loro vita con la mia lasciandomi un insegnamento, una risata, un'emozione: grazie per essere stati parte di questi otto lunghi, intensi ed entusiasmanti anni.

\bigskip\noindent
Quasi dimenticavo: grazie geco per la tua vicinanza poco prima del mio ultimo esame, sei stato un terrificante portafortuna. 

\bigskip\noindent
Grazie, infine --questa volta per davvero-- al mio anaffettivo quanto esuberante gatto Joule per essere stato mio fedele compagno di studi e per i sorrisi che mi hai strappato tutte le volte che distoglievo lo sguardo dai libri e ti trovavo comodamente seduto, come il più ordinario degli umani. Anche se non sai leggere, grazie per avermi rubato il cuore e per avermi insegnato, con la tua superiorità felina, di amare sempre prima se stessi.  