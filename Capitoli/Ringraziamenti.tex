\setcounter{secnumdepth}{-1}
\chapter{Ringraziamenti}
%\addtocentrydefault{chapter}{}{Ringraziamenti}
%\addcontentsline{toc}{chapter}{Ringraziamenti}

Questo spazio lo dedico alle persone che, con il loro supporto, mi hanno aiutato in questo lungo quanto arricchente percorso universitario. 

\bigskip \noindent
Ringrazio innanzitutto il mio relatore, prof. Barbi, per l'impareggiabile disponibilità e per avermi trasmesso lezioni che non si possono apprendere dai libri.

\bigskip \noindent
Un ringraziamento speciale alla mia correlatrice, dott.ssa Martina D'Agostin, che ha saputo guidarmi e consigliarmi con pazienza ed entusiasmo nella realizzazione di quest'elaborato. %e lasciando spazio alle mie idee. 

\bigskip \noindent
Al mio ragazzo Gabriele: grazie per esserci stato sempre e per avermi ascoltata ripetere con ironica gioia tutte le materie: da anatomia in poi. Finalmente posso confessare di averti segretamente addestrato in modo da poter contare su di te ogni qualvolta avrò dei dubbi sulla gestione di un paziente. Grazie, poi, per non aver creduto nemmeno una volta che non avrei superato un esame, anche se continuavo a ribadirti che sarebbe sicuramente successo. Grazie perché quando, invece, è accaduto e/o quando si sono spente le luci e vedevo questo obiettivo come gigante e irraggiungibile, tu mi hai spronata a trasformare la paura in coraggio e ad invertire l'incantesimo pronunciando Lumos Maxima. Inoltre --la tua parola preferita--, grazie per avermi spinta ad essere più sicura, consapevole e fiduciosa in me stessa: continuerò comunque a chiederti di leggermi le mail prima di inviarle. Passando, infine, ai ringraziamenti più concreti, non posso non mostrarti la mia infinita gratitudine per aver condiviso con me le tue nozioni di programmazione ed avermi pazientemente aiutato con i contenuti grafici e l'analisi statistica. Senza di te e la tua inarrestabile pignolaggine questo lavoro di tesi non sarebbe altrettanto ben fatto. 

%Grazie perché quando immaginavo questo obiettivo come immenso e irraggiungibile. Grazie per avermi insegnato ad avere coraggio, nonostante la paura. Grazie per essere ed essere stato in questi anni il mio compagno di avventure  e il mio migliore amico, semplicemente la mia persona e grazie per avermi dato l'opportunità di arricchire il mio bagaglio di conoscenze, idee, opinioni, grazie alle nostre discussioni e ai nostri confronti sui temi più svariati. Grazie perché se sono orgogliosa della persona che sono oggi è anche merito tuo.  Infine, 

\bigskip \noindent
Ringrazio i miei genitori per avermi sempre sostenuto, supportato e per avermi permesso di portare a termine gli studi universitari, aiutandomi ad affrontare i momenti più difficili. 

\bigskip \noindent
Ringrazio mia sorella Viola per avermi mostrato che dalla costanza e dall'impegno nascono i migliori risultati ma soprattutto per i post-it motivazionali. Non li dimenticherò. 

\bigskip \noindent
Grazie alla mia nonnina Ester, che ha impazientemente atteso questo traguardo e che è quasi più emozionata di me nel vedermi, finalmente, conquistarlo.

\bigskip \noindent
Grazie alle mie coinquiline Fra e Sere che hanno condiviso con me le gioie e i topi di questo percorso. Grazie per aver sopportato le mie pizze bruciate e lo scotch sugli occhiali. Non sarebbero potute capitarmi migliori compagne d'avventura, vi voglio bene e vi avrò sempre nel cuore. 

\bigskip \noindent
Un grazie immenso ai miei amici Giulia N., Giulia O. e Andrea, che sono sempre stati pronti ad ascoltarmi ed aiutarmi ogni qualvolta una nuova difficoltà si è presentata in questi ultimi mesi.

\bigskip \noindent
Grazie ad Ariettis per aver reso da subito spumeggiante e vivernosa quest'esperienza e, quando il gioco si è fatto duro, avermi spronato a buttarmi: senza i tuoi incitamenti adesso starei ancora preparando fisiopato.

\bigskip \noindent
Grazie a Jimmy che, dopo avermi fatto da colonna sonora per tutto il liceo, è rimasto il mio inamovibile compagno di banco, nonostante i 170km di distanza. Grazie per aver ascoltato ogni mio singolo audio di 8 minuti e grazie per essere stato il primo ad avermi fatto sentire un "medico": spero che la qualità dei miei consigli migliorerà sempre di più. 

\bigskip\noindent
Infine, grazie a tutti i miei amici vicini e lontani e a tutti quelli che hanno incrociato la loro vita con la mia, lasciandomi un insegnamento, una risata, un'emozione: grazie per essere stati parte di questi otto lunghi, intensi ed entusiasmanti anni.

\bigskip\noindent
Quasi dimenticavo! Grazie geco per la tua vicinanza a ridosso del mio ultimo esame: sei stato un viscido portafortuna. 

\bigskip\noindent
Grazie, infine --questa volta per davvero--, al mio anaffettivo quanto esuberante gatto Joule per essere stato mio fedele compagno di studi e per i sorrisi che mi hai strappato tutte le volte che distoglievo lo sguardo dai libri e ti trovavo comodamente seduto, come il più ordinario degli umani. Anche se non sai parlare, grazie per avermi rubato il cuore e per avermi insegnato, con la tua superiorità felina, di amare sempre prima se stessi.  