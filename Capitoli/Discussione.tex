\chapter{Discussione}
%\`E

Questo lavoro si basa sulla convinzione che l'opinione infermieristica sia un indicatore fondamentale per valutare la qualità del servizio di cura offerto quotidianamente ai pazienti pediatrici che si sottopongono a procedure diagnostiche o terapeutiche. Da questo presupposto nasce la volontà di studiare e confrontare le considerazioni e le critiche di una categoria professionale che, oltre ad essere parte attiva durante tutto il percorso di sedazione, spesso riconosce e risponde per prima ai bisogni fisici e psicologici dei piccoli pazienti. Le prospettive di quest'area di ricerca, tutt'ora poco esplorata, stanno nella possibilità di implementare nella pratica medica quotidiana i risultati emersi, agendo sia, laddove possibile, sulla scelta farmacologica, sia sulla comunicazione ed il potenziamento del lavoro di squadra tra medici ed infermieri, garantendo ai pazienti e alle famiglie un'assistenza sempre di maggior qualità e successo. 



%In particolare, in questo studio, lo scopo del raccogliere le opinioni degli infermieri risiede nella volontà di analizzare le considerazioni e le critiche di una categoria professionale che, oltre ad essere parte attiva durante tutto il percorso di sedazione, spesso riconosce e risponde per prima ai bisogni fisici e psicologici dei piccoli pazienti. L'interesse in quest'area di ricerca, tutt'ora poco esplorata, sta nella possibilità di implementare nella pratica medica quotidiana i risultati emersi, agendo sia, laddove possibile, sulla scelta farmacologica, sia sulla comunicazione ed il potenziamento del lavoro di squadra tra medici ed infermieri, garantendo ai pazienti e alle famiglie un'assistenza sempre di maggior qualità e successo. 
\section{Preferenze farmacologiche ed effetti avversi}

\subsection*{Il propofol è stato il farmaco più apprezzato}
Dalla presente analisi emerge che circa metà degli infermieri intervistati preferisce il propofol agli altri agenti farmacologici testati per le sedazioni procedurali. Questo dato risulta in linea con alcuni studi, in cui il propofol, comparato ad altri regimi farmacologici, si rivela un agente altrettanto efficace ma con dei tempi di recupero più brevi e delle caratteristiche di risveglio più favorevoli \citep{Schacherer2019, Havel1999, Shah2011}. Inoltre, il punteggio di soddisfazione associato al propofol aumenta con il numero di sedazioni mensilmente effettuate.

\subsection*{La ketamina ha ottenuto i valori di soddisfazione più bassi}
Il livello di soddisfazione complessivo percepito dai partecipanti si è delineato ugualmente elevato per il propofol, per il midazolam e per la dexmedetomidina, mentre è risultato significativamente inferiore per la ketamina. Gli elementi che hanno influito di più su questo giudizio sono stati la qualità del risveglio e gli effetti avversi: i più frequentemente riportati dagli infermieri sono gli stessi già ampiamente conosciuti in letteratura \citep{Bellolio2016, Krauss2006}. Tra questi il distress respiratorio, le allucinazioni, la nausea e il vomito, le vertigini, l'irritabilità e l'iperattività sono gli effetti collaterali che hanno avuto una maggior influenza sul punteggio complessivo della qualità della sedazione. Alla luce di ciò, è interessante notare come la nausea e il vomito, le allucinazioni, l'irritabilità e l'iperattività coincidano con le reazioni avverse più frequentemente indicate dagli infermieri relativamente alla sedazione con la ketamina\footnote{Il distress respiratorio non è stato riscontrato dagli infermieri per nessuno dei quattro agenti farmacologici testati.}. Inoltre, essa è stata associata anche ad una maggior necessità di farmaci \emph{rescue} o sintomatici, ad esempio antiemetici.
Nonostante la ketamina sia considerata gold standard dai pediatri per la sua efficacia e l'elevata sicurezza \citep{Krauss2006}, l'opinione degli infermieri mette in evidenza alcuni limiti, per lo più relativi alle caratteristiche del risveglio. Un'alternativa possibile è rappresentata dal propofol, il quale possiede un sovrapponibile profilo di sicurezza ed efficacia per quanto concerne le sedazioni procedurali minimamente e moderatamente dolorose \citep{Vardi2002, Ferguson2016, Jalili2016}; inoltre, è stato correlato principalmente a sonnolenza, un effetto avverso di minor impatto sulla percezione infermieristica. In conclusione, quest'indagine conferma l'importanza di una valutazione che tenga conto delle opinioni dei professionisti di tutta l'équipe, al fine di definire le strategie farmacologiche migliori per ogni paziente.

\section{Livello di sicurezza percepito per i quattro agenti farmacologici}
Gli intervistati hanno riferito di sentirsi molto sicuri durante le sedazioni con tutti e quattro gli agenti farmacologici, nonostante queste non siano prive di rischi \citep{Bellolio2016}. Questo risultato dovrebbe incoraggiare l'uso sistematico di sedativi ed analgesici nei bambini che devono sottoporsi a procedure diagnostiche o terapeutiche, purché la sedazione avvenga in setting adeguati, con personale appropriatamente formato e sotto un attento monitoraggio. 

\subsection*{Midazolam vs dexmedetomidina}
Il midazolam è stato indicato dagli infermieri come farmaco più sicuro rispetto alla dexmedetomidina. Nonostante il valore di significatività statistica di tale risultato sia molto vicino ad $\alpha$ (p--value 0.048) e quindi poco rilevante dal punto di vista statistico, ci sono alcune ragioni che potrebbero sottostare a questa risposta. Nel centro in cui è stato svolto questo lavoro di tesi, il midazolam viene utilizzato prevalentemente a dosaggi bassi, al fine di ottenere il solo effetto ansiolitico: ad esempio, per ridurre l'ansia preprocedurale e facilitare il posizionamento dell'accesso venoso periferico. Il rischio di depressione respiratoria associato alla somministrazione del midazolam è dose correlato e, quindi, limitato con tale posologia, inoltre eventuali reazioni avverse possono essere rapidamente annullate grazie al flumazenil, un antagonista delle benzodiazepine. La dexmedetomidina, invece, è indubbiamente un farmaco con un elevato profilo di sicurezza \citep{Sulton2016} e per tale ragione, oltre che per il livello di efficacia, è stato considerato da alcuni studi superiore al midazolam \citep{Barends2017, Lin}. Tuttavia, andando ad analizzare il livello di soddisfazione dei familiari rispetto alla sedazione con questo farmaco si possono fare alcune considerazioni. La dexmedetomidina viene frequentemente utilizzata nei pazienti più piccoli per studi radiologici lunghi, quali la RM. L'ambiente ristretto in cui si svolge tale esame, in un lavoro, ha influenzato negativamente il grado di soddisfazione parentale \citep{Lew2010}. Inoltre, può presentare tempi di recupero più lunghi di altri agenti farmacologici: questo elemento, il tipo di procedura e l'età più bassa dei bambini sottoposti a questo tipo di sedazione ha portato, secondo gli autori di un altro studio \citep{Cortellazzo2022}, la combinazione dexmedetomidina + midazolam, ad ottenere, dai genitori, dei punteggi di soddisfazione inferiori rispetto agli altri regimi farmacologici, seppur complessivamente elevati.


%sulla soddisfazione del caregiver, riportato estesamente nella sezione 1.1., come possibili fattori determinanti il punteggio relativ  pediatriche questi motivi la combinazione di dexmedetomidina e midazolam ha ricevuto i voti più bassi anche nello studio relativo alla soddisfazione parentale in merito alle sedazioni procedurali pediatriche \citep{Cortellazzo2022}, riportato nella sezione 1.1.  

%La ketamina è considerato un gold standard dai pediatri a causa della sua efficacia e sicurezza. Tuttavia, nel contesto delle sedazioni procedurali da minimamente a moderatamente dolorose il propofol potrebbe essere altrettanto efficace e sicuro. Quest'evidenza suggerisce che l'opinione degli infermieri dovrebbe essere condivisa/appoggiata/sostenuta nell'ambito di un lavoro di squadra nel definire una strategia di sedazione per specifico paziente. La soddisfazione infermieristica per il propofol aumenta con il numero di sedazioni eseguite per mese. 

\section{Vie di somministrazione preferite e tecniche di distrazione}
Le vie di somministrazione preferite dal personale infermieristico sono risultate la via endovenosa e la via intranasale più la via orale. Oltre a ciò, il 74.5$\%$ degli infermieri ha giudicato le tecniche di distrazione come importanti o molto importanti. Quest'ultime sono semplici, economiche, facili da imparare e non richiedono tempi di applicazione eccessivi. Precedenti studi hanno dimostrato l'efficacia dell'approccio non farmacologico, da solo o in combinazione al trattamento farmacologico \citep{Tibaldo2020, Koller2012}: permette, infatti, di controllare il dolore associato a interventi minimamente dolorosi, come le punture venose, di distogliere l'attenzione del bambino dalla procedura, di ridurre la paura e, quindi, di favorire la collaborazione. La consapevolezza degli infermieri in merito alla rilevanza delle tecniche di distrazione rappresenta per tutti i medici un solido promemoria dell'importanza di questa centrale componente della cura. 
%La via EV e la IN + OS sono le preferite da questa ricerca. Il 74.5 degli intervistati ha votato le tecniche di distrazione come importanti. Infatti, sono semplice, economiche, facili da imparare, non dispendiose in termini di tempo. Precedenti studi hanno visto che gli interventi non farmacologici nei pazienti pediatrici riducono il dolore associato a procedure minimamente dolorose quali venipunture, e hanno mostrato che queste tecniche, da sole o in combinazione con il trattamento farmacologico possono ridurre il dolore e la paura, oltre a promuovere la collaborazione di bambino e genitori. La consapevolezza degli infermieri rispetto alla rilevanza delle tecniche non farmacologiche e un adeguata comunicazione con pazienti e famiglie è un forte promemoria per tutti i medici dell'importanza di questa determinante/cruciale/centrale/decisivo componente di cura. 

%Indipendentemente dal regime farmacologico, hanno riportato di sentirsi sicuri durante le sedazioni procedurali anche se non sono completamente prive di rischio (cit). Questo risultato dovrebbe incoraggiare l'uso sistematico dei sedativi nei setting adeguati con definiti livelli di training e monitoraggio per i bambini che si sottopongono a procedure diagnostiche o terapeutiche che causano dolore o stress eccessivo. Gli eventi avversi più frequentemente riportati sono quelli già conosciuti in letteratura (cit.) Distress respiratorio, allucinazioni e nausea/vomito hanno avuto un impatto maggiore. 

\section{Limiti dello studio}
Questo lavoro ha alcune limitazioni. Innanzitutto, è stato utilizzato un questionario non validato, dal momento che non è disponibile in letteratura uno strumento validato per raccogliere i dati necessari agli scopi di questa ricerca. Secondariamente, il numero di infermieri intervistati è relativamente ridotto poiché l'indagine è stata limitata ad un singolo centro. Inoltre, molti infermieri potrebbero non aver avuto esperienza con tutti i tipi di sedativi ed analgesici testati. 

\section{Punti di forza dello studio}
Gli elementi di robustezza di quest'analisi consistono sia nell'arruolamento di infermieri pediatrici con rilevante esperienza nel campo delle sedazioni procedurali, sia nella realizzazione in due tempi del questionario, la cui versione definitiva è frutto del contributo congiunto di medici, infermieri e genitori. 