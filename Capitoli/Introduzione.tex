\chapter{Introduzione}

Negli ultimi decenni, anche in ambito pediatrico, si è osservato un aumento dell'esecuzione di procedure diagnostiche o terapeutiche, che non necessitano dell'utilizzo della sala operatoria. Queste procedure possono essere dolorose e/o richiedere che l'individuo rimanga immobile per diverse decine di minuti, ragion per cui nel bambino, posto che in tenera età non si possiedono le strutture mentali adeguate ad affrontare e comprendere l'evento, possono essere vissute come un'esperienza stressante e negativa.\footnote{D'altro canto è ben documentato \cite{Memoriadolorechir, Memoriadol2004, Memoryforpain1999} che ciò può determinare una \emph{memoria del dolore}, con possibili implicazioni cliniche e/o una significativa riduzione della compliance in successive esperienze in ambito medico-ospedaliero.}
\\Per aggirare tali problematiche viene utilizzata la sedazione procedurale \cite{Krauss2006, Simeupsedazione}, ossia uno stato medicalmente indotto di depressione della coscienza, con mantenimento della funzionalità cardiorespiratoria. Viene operata prima e durante procedure diagnostiche o terapeutiche di breve durata, da figure professionali qualificate e adeguatamente preparate, benché senza un profilo professionale strettamente anestesiologico. 
\\Con le conoscenze odierne, al fine di migliorare l'outcome clinico dei pazienti e, più in generale la qualità del servizio sanitario offerto, risulta utile analizzare e confrontare le caratteristiche del risveglio post sedazione procedurale, che dipendono, oltre che da fattori soggettivi, anche da variabili oggettive quali la scelta dei farmaci utilizzati. 
\\In questo lavoro di tesi verranno messi a confronto quattro dei regimi farmacologici più comunemente utilizzati, valutando la qualità del risveglio come percepita dal personale infermieristico, punto di riferimento per i genitori e i piccoli pazienti nelle fasi di recupero post sedazione. Inoltre, verranno riferiti gli effetti collaterali più comunemente osservati e la loro frequenza. L'obiettivo è di fornire al sedatore delle indicazioni aggiuntive, fra le tante, che determinano la scelta del regime farmacologico da utilizzare.

\section{Background}

Sebbene la sedazione per le procedure diagnostiche o terapeutiche sia pratica quotidiana in ambito pediatrico, esiste ancora poca letteratura focalizzata sulla qualità del risveglio post sedazione procedurale. Esistono, invece, diversi studi sistematici su frequenza e insorgenza di effetti avversi, in relazione ai diversi farmaci utilizzati \citep{Bellolio2016}. 

\subsection{A subsection}

\lipsum[3]

\subsubsection{A subsubsection}

\lipsum[4]