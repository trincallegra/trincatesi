\chapter{Introduzione}

%\`E

Negli ultimi decenni, anche in ambito pediatrico, si è osservato un aumento dell'esecuzione di procedure diagnostiche o terapeutiche, che non necessitano dell'utilizzo della sala operatoria. Queste procedure possono essere dolorose e/o richiedere che l'individuo rimanga immobile per diverse decine di minuti, ragion per cui nel bambino, posto che in tenera età non si possiedono le strutture mentali adeguate ad affrontare e comprendere l'evento, possono essere vissute come un'esperienza stressante e negativa\footnote{D'altro canto è ben documentato \cite{Memoriadolorechir, Memoriadol2004, Memoryforpain1999} che ciò può determinare una \emph{memoria del dolore}, con possibili implicazioni cliniche e/o una significativa riduzione della compliance in successive esperienze in ambito medico-ospedaliero.}.
\\Per aggirare tali problematiche viene utilizzata la sedazione procedurale, ossia uno stato medicalmente indotto di depressione della coscienza, con mantenimento della funzionalità cardiorespiratoria. Viene operata prima e durante procedure diagnostiche o terapeutiche di breve durata, da figure professionali qualificate e adeguatamente preparate, benché senza un profilo professionale strettamente anestesiologico \cite{Krauss2006, Simeupsedazione}. 
\\Con le conoscenze odierne, al fine di migliorare l'outcome clinico dei pazienti e, più in generale l'offerta del servizio sanitario, risulta utile analizzare la qualità della sedazione procedurale e confrontare le caratteristiche del risveglio successivo, che dipendono, oltre che da fattori soggettivi, anche da variabili oggettive quali la scelta dei farmaci utilizzati. 
\\In questo lavoro di tesi verranno messi a confronto quattro dei regimi farmacologici più comunemente utilizzati, valutando la percezione del personale infermieristico, punto di riferimento per i genitori e i piccoli pazienti nelle fasi antecedenti di ammissione e ricovero e in quelle di recupero post sedazione. Inoltre, verranno riferiti gli effetti collaterali più comunemente osservati e la loro influenza sul giudizio di soddisfazione complessivo. L'obiettivo è di fornire al sedatore delle indicazioni aggiuntive, fra le tante che determinano la scelta del regime farmacologico da utilizzare.

\section{Background}

Sebbene la sedazione per le procedure diagnostiche o terapeutiche sia pratica quotidiana in ambito pediatrico, esiste ancora poca letteratura focalizzata sulla qualità della sedazione e del risveglio seguente. Sono noti e ben documentati \citep{Bellolio2016, Hartlin2016}, invece, i profili di sicurezza ed efficacia dei farmaci disponibili per tale finalità. Una panoramica sulle caratteristiche farmacologiche, indicazioni, posologia ed effetti avversi dei più comuni agenti sedativi è riportata nell'appendice B. %Esistono, invece, diversi studi sistematici su frequenza e insorgenza di effetti avversi, in relazione ai diversi farmaci utilizzati

\bigskip

\subsection*{Soddisfazione parentale nella sedazione procedurale pediatrica}
Nonostante le caratteristiche del risveglio post sedazione procedurale siano state finora raramente al centro dell'indagine scientifica, recentemente è stato pubblicato uno studio focalizzato sulla qualità del risveglio post procedurale, valutata da parte del caregiver \cite{Cortellazzo2022}. %In questo articolo sono anche stati esposti i fattori più frequentemente associati ad un basso livello di gradimento, che tuttavia si è complessivamente dimostrato elevato per tutti i regimi farmacologici confrontati.  

\begin{description}
\item[Metodi] 

%\subsubsection*{Metodi} 
Si tratta di uno studio osservazionale prospettico, condotto presso l'ospedale universitario materno infantile IRCSS Burlo Garofolo di Trieste, in cui sono stati confrontati quattro regimi farmacologici ---propofol, propofol più midazolam (\texttt{EV} o \texttt{OS}), ketamina più propofol, dexmedetomidina (\texttt{IN}) più midazolam (\texttt{OS})---\footnote{Legenda: \texttt{OS} = per via orale, \texttt{EV} = per via endovenosa, \texttt{IN} = per via intranasale.} utilizzati per la sedazione procedurale di bambini tra gli 0 e i 18 anni. La scelta del farmaco si è basata su diverse variabili, quali l'età del paziente, il livello di sedazione desiderato ed il tipo di procedura, oltre alla necessità di analgesia ed alla storia di eventi avversi in precedenti sedazioni. 
\\Allo scopo dello studio è stato sviluppato un questionario non validato\footnote{Al momento in letteratura non esistono linee guida per misurare il livello di soddisfazione in merito alla sedazione pediatrica.}, in cui è stato chiesto ai genitori dei piccoli pazienti arruolati nel lavoro di ricerca (in totale 655) di valutare il grado di soddisfazione della sedazione procedurale con un punteggio \texttt{NRS}\footnote{\emph{Numerical Rating Scale}, da 0 a 10.}, in due fasi: nel periodo intercorso dalla fine della procedura al completo risveglio del bambino ed il giorno dopo la procedura. In questo modo è stato possibile indagare sia la presenza di effetti avversi e la percezione del risveglio dal punto di vista del caregiver, sia la presenza di sintomi tardivi ed il grado di soddisfazione complessivo dell'esperienza di sedazione.

\item[Risultati] 
%\subsubsection*{Risultati}
Il grado di soddisfazione percepito dal caregiver durante la prima fase è stato molto elevato: ben il 77.7$\%$ degli intervistati ha dato una valutazione \texttt{NRS} $\geq$ 8, senza differenze significative tra il gruppo di soggetti sottoposti a procedure dolorose e quello in cui l'analgesia non era necessaria. In particolare, un livello \texttt{NRS} $\geq$ 8 è stato riscontrato nel 81.2$\%$ dei pazienti che hanno ricevuto solo propofol, nel 80.8$\%$ di coloro che hanno ricevuto propofol + midazolam, nel 79.1$\%$ del gruppo ricevente ketamina + propofol, mentre i livelli di soddisfazione più bassi sono stati osservati con la combinazione di dexmedetomidina + midazolam (66.1$\%$). Tuttavia, stratificando per l'età, in particolar modo restringendo l'analisi alla fascia d'età 0-4 anni, le differenze non risultano statisticamente significative. Lo stesso quadro viene confermato per la seconda fase, in cui viene valutata complessivamente l'esperienza: è stata infatti giudicata con un livello di gradimento altrettanto elevato: l'82.3$\%$ dei caregiver ha dato un punteggio \texttt{NRS} $\geq$ 8. Anche in questo caso l'associazione dexmedetomidina + midazolam ha mostrato gradi di soddisfazione più bassi (68.5$\%$) rispetto agli altri agenti farmacologici.
\\Gli effetti avversi più comunemente riscontrati dopo il risveglio sono: sonnolenza (55.4$\%$) ed irritabilità (24.6$\%$), seguiti meno frequentemente da irrequietezza ed agitazione, instabilità e vertigini, cefalea, alterazioni nell'appetito, nausea o vomito e molto raramente allucinazioni e distress respiratorio. Ad eccezione delle allucinazioni, non rilevate come sintomo tardivo, il giorno dopo la sedazione sono state riportate da alcuni genitori le precedentemente elencate reazioni avverse nello stesso ordine di frequenza, pur con un'incidenza molto più bassa. In questa ricerca l'insorgenza di effetti collaterali ha influenzato negativamente il grado di soddisfazione parentale ed è stata più frequentemente associata ai bambini che hanno ricevuto combinazioni di farmaci piuttosto che a coloro a cui è stato somministrato solo il propofol. Anche il tempo di recupero e di ripresa delle normali attività si è dimostrato essere più rapido (circa 1 ora) nei soggetti che hanno ricevuto solo il propofol, mentre è stato più lento (circa 3 ore) nei soggetti riceventi dexmedetomidina + midazolam. 
\item[Discussione] 
%\subsubsection*{Discussione}
Questo studio ha dimostrato un elevato livello di soddisfazione parentale relativa alla sedazione pediatrica, indipendentemente dal regime farmacologico utilizzato e dalla presenza o meno di dolore associato alla procedura. I principali fattori che hanno influenzato negativamente il giudizio genitoriale sono legati alla comparsa di eventi avversi ed all'età del bambino: infatti i caregiver dei pazienti più piccoli hanno valutato l'esperienza di sedazione con punteggi più bassi.
Alla luce di ciò si possono giustificare i valori di gradimento inferiori associati alla somministrazione di dexmedetomidina + midazolam, combinazione frequentemente utilizzata nei bambini più piccoli, sottoposti a procedure di maggior durata (ad esempio RM). Oltretutto, è stata associata a tempi di risveglio più lunghi e ad una maggior incidenza di episodi di irritabilità. 

\item[Limiti]
%\subsubsection*{Limiti}
I principali limiti di questo studio riguardano la comprensibile inclinazione del genitore ad includere nel metro di giudizio diversi fattori, non strettamente legati alle caratteristiche del risveglio post sedazione e a valutare l'esperienza ospedaliera nel complesso. Ad esempio, alcuni elementi oggetto di bias possono essere la gravità della patologia del figlio, il tipo e la durata della procedura, precedenti esperienze di sedazione, il background culturale e l'interazione con il personale. 
\\Inoltre, poiché la scelta del farmaco è stata effettuata caso per caso, in base all'età ed al tipo di procedura, un altro limite di questa ricerca consiste nell'aver comparato gruppi farmacologici non omogenei in termini di numerosità ed età dei pazienti.
\item[Prospettive] 
%\subsubsection{Prospettive}

In conclusione, gli autori dello studio suggeriscono l'utilità di ulteriori indagini e ricerche al fine di confermare questi risultati e per discernere le componenti soggettive ed oggettive correlate alla valutazione della qualità del risveglio post procedurale. Questo lavoro di tesi si propone di rispondere, almeno parzialmente, a tale esigenza.


\end{description}



\section{Scopi}

L'indagine, che verrà ampiamente illustrata in seguito, si immerge nell'ambito delle sedazioni procedurali pediatriche, campo in evoluzione e crescente diffusione in tutto il mondo ma il cui aspetto qualitativo risulta tutt'ora inesplorato dalla comunità scientifica.
%dal punto di vista della qualità dell'esperienza di sedazione.
Per tale ragione questo lavoro mira a valutare il livello di soddisfazione del personale infermieristico associato ai diversi regimi farmacologici utilizzati durante le sedazioni effettuate fuori dalla sala operatoria. Nello specifico verranno testate la percezione del profilo di sicurezza dei farmaci, le preferenze relative al sito di somministrazione, la frequenza d'insorgenza di eventuali effetti avversi e il giudizio relativo alla qualità della sedazione nel suo complesso.
Inoltre, i risultati di questo lavoro verranno confrontati con l'esigua letteratura nota, in particolare con lo studio sopra descritto \cite{Cortellazzo2022}, in cui è stata scelta la figura del caregiver come indicatore di qualità della sedazione, con tuttavia alcuni limiti. L'obiettivo finale è quello di fornire delle indicazioni complete ed oggettive ai sedatori, affinché le possano applicare, laddove possibile, nella scelta farmacologica, con lo scopo di offrire al paziente ed alla famiglia uno standard di cura ed una qualità del servizio sanitario sempre maggiori.

%I principali obiettivi di quest'indagine concernono la volontà di fornire indicazioni quanto più oggettive in merito alla qualità della sedazione procedurale e del risveglio successivo. Mira, quindi, a mettere a confronto i quattro farmaci sedativi, analgesici e dissociativi più ampiamente utilizzati, avvalendosi della percezione infermieristica come indicatore di soddisfazione. 
%tuttavia rimane ancora inesplorato dalla comunità scientifica bensì, tuttavia, riveste un ruolo di rilevo sia  considerata la sempre maggiore diffusione e pratica in tutto il mondo della sedazione procedurale sia tenuta presente la volontà, in un'ottica di progresso, 
%Quest'indagine si immerge in un campo ancora inesplorato dalla comunità scientifica bensì, tuttavia, riveste un ruolo di rilevo sia  considerata la sempre maggiore diffusione e pratica in tutto il mondo della sedazione procedurale sia tenuta presente la volontà, in un'ottica di progresso, di migliorare le prestazioni offerte: 

%\subsection{A subsection}

%\lipsum[3]

%\subsubsection{A subsubsection}

%\lipsum[4]