\chapter{Introduzione}

%\`E

Negli ultimi decenni, anche in ambito pediatrico, si è osservato un aumento dell'esecuzione di procedure diagnostiche o terapeutiche, che non necessitano dell'utilizzo della sala operatoria. Queste procedure possono essere dolorose e/o richiedere che l'individuo rimanga immobile per diverse decine di minuti, ragion per cui nel bambino, posto che in tenera età non si possiedono le strutture mentali adeguate ad affrontare e comprendere l'evento, possono essere vissute come un'esperienza stressante e negativa\footnote{D'altro canto è ben documentato \cite{Memoriadolorechir, Memoriadol2004, Memoryforpain1999} che ciò può determinare una \emph{memoria del dolore}, con possibili implicazioni cliniche e/o una significativa riduzione della compliance in successive esperienze in ambito medico-ospedaliero.}.
\\Per aggirare tali problematiche viene utilizzata la sedazione procedurale, ossia uno stato medicalmente indotto di depressione della coscienza, con mantenimento della funzionalità cardiorespiratoria. Viene operata prima e durante procedure diagnostiche o terapeutiche di breve durata, da figure professionali qualificate e adeguatamente preparate, benché senza un profilo professionale strettamente anestesiologico \cite{Krauss2006, Simeupsedazione}. 
\\Con le conoscenze odierne, al fine di migliorare l'outcome clinico dei pazienti e, più in generale l'offerta del servizio sanitario, risulta utile analizzare la qualità della sedazione procedurale e confrontare le caratteristiche del risveglio successivo, che dipendono, oltre che da fattori soggettivi, anche da variabili oggettive quali la scelta dei farmaci utilizzati. 
\\In questo lavoro di tesi verranno messi a confronto quattro dei regimi farmacologici più comunemente utilizzati, valutando la percezione del personale infermieristico, punto di riferimento per i genitori e i piccoli pazienti nelle fasi di ammissione e ricovero e in quelle di recupero post sedazione. Inoltre, verranno riferiti gli effetti collaterali più comunemente osservati e la loro influenza sul giudizio di soddisfazione complessivo. L'obiettivo è di fornire al sedatore delle indicazioni aggiuntive, fra le tante che determinano la scelta del regime farmacologico da utilizzare.



\bigskip
\begin{comment}


\section{Background}

Sebbene la sedazione per le procedure diagnostiche o terapeutiche sia pratica quotidiana in ambito pediatrico, esiste ancora poca letteratura focalizzata sulla qualità della sedazione e del risveglio seguente. Sono noti e ben documentati \citep{Bellolio2016, Hartlin2016}, invece, i profili di sicurezza ed efficacia dei farmaci disponibili per tale finalità. Una panoramica sulle caratteristiche farmacologiche, indicazioni, posologia ed effetti avversi dei più comuni agenti sedativi è riportata nell'appendice B. %Esistono, invece, diversi studi sistematici su frequenza e insorgenza di effetti avversi, in relazione ai diversi farmaci utilizzati

\end{comment}
\newpage

\section{Soddisfazione parentale ed infermieristica nella sedazione procedurale pediatrica}

In ambito pediatrico, la soddisfazione infermieristica e parentale rappresenta un indicatore di qualità delle cure offerte. Conoscere, quindi, il giudizio dei caregiver e degli infermieri in merito alla sedazione procedurale può essere utile nel guidare le scelte del sedatore verso la migliore opzione farmacologica per ciascun bambino. Nonostante ciò, fino ad oggi, il punto di vista dei genitori e del personale infermieristico sulla qualità della sedazione svolta al di fuori del contesto operatorio, è stata raramente al centro dell'indagine scientifica. In uno studio effettuato presso un'unità di terapia intensiva pediatrica sono stati intervistati medici ed infermieri prima e dopo l'implementazione di nuove linee guida per la gestione ed il trattamento di dolore, agitazione e delirio, frequenti nei bambini critici e responsabili di possibili implicazioni fisiche e psicologiche. \`E emerso, in seguito all'applicazione delle nuove indicazioni, un miglioramento generale della collaborazione e della comunicazione tra i professionisti impiegati nell'unità intensiva pediatrica, con risvolti positivi sulla cura ed il benessere dei pazienti \citep{Staveski2017}. Oltre a ciò, esistono in letteratura alcuni lavori focalizzati sulla percezione dell'esperienza sedativa valutata da parte del caregiver. I genitori e i tutori legali dei bambini sottoposti a sedazioni procedurali sono complessivamente risultati molto soddisfatti dell'esperienza in tutti gli studi, indipendentemente dagli agenti farmacologici utilizzati \citep{Connor2014, Lew2010, Cortellazzo2022}. I principali elementi che hanno influenzato negativamente il giudizio parentale sono stati: la presenza di effetti avversi al risveglio, una lunga durata della sedazione, un ambiente non a misura di bambino o poco spazioso, eventuali problemi di comunicazione e il mancato rispetto delle tempistiche \citep{Lew2010, Connor2014}.
In particolare, recentemente è stato pubblicato uno studio incentrato sulla qualità del risveglio post procedurale, valutata da parte del caregiver \citep{Cortellazzo2022}. A differenza dei precedenti, questo lavoro mette a confronto gli agenti farmacologici attualmente più utilizzati dai sedatori ed analizza la soddisfazione parentale relativa alle sedazioni procedurali svolte dalla stessa Unità Operativa Pediatrica protagonista di questo lavoro di tesi. Si tratta di uno studio osservazionale prospettico, condotto presso l'ospedale universitario materno infantile IRCSS Burlo Garofolo di Trieste, che ha coinvolto 655 bambini: è stato chiesto ai genitori di fornire una valutazione sul proprio grado di soddisfazione relativo alla sedazione offerta al figlio. Complessivamente, il giudizio dei caregiver è risultato molto positivo per tutti i regimi farmacologici utilizzati, senza differenze statisticamente significative tra i quattro testati: propofol, propofol più midazolam (\texttt{EV} o \texttt{OS}), ketamina più propofol, dexmedetomidina (\texttt{IN}) più midazolam (\texttt{OS})\footnote{Legenda: \texttt{OS} = per via orale, \texttt{EV} = per via endovenosa, \texttt{IN} = per via intranasale.}. I principali fattori che hanno influenzato negativamente il giudizio genitoriale sono legati alla comparsa di eventi avversi ed all’età del bambino: infatti, i caregiver dei pazienti più piccoli hanno valutato l’esperienza di sedazione con punteggi più bassi. I principali limiti di questo studio riguardano la comprensibile inclinazione del genitore ad includere nel metro di giudizio diversi fattori, non strettamente legati alle caratteristiche del risveglio post sedazione e a valutare l’esperienza ospedaliera nel complesso. Ad esempio, alcuni elementi oggetto di bias possono essere la gravità della patologia del figlio, il tipo e la durata della procedura, precedenti esperienze di sedazione, il background culturale e l’interazione con il personale. 

\begin{comment}

%non esistono dati specifici sulla percezione infermieristica riguardo la qualità della sedazione svolta al di fuori del teatro operatorio.
%Di seguito verranno esposti in dettaglio i metodi, gli esiti e i limiti di tale studio, al fine di compararli successivamente con i risultati emersi da questo progetto di tesi.

%La soddisfazione del paziente e della famiglia è una componente fondamentale per garantire un'elevata qualità di cura. Infatti, un'esaustiva comunicazione ed un'adeguata relazione di fiducia tra medico, caregiver e paziente permette di raggiungere migliori outcome clinici.

%Nonostante l caratteristiche del risveglio post sedazione procedurale siano state finora raramente al centro dell'indagine scientifica, recentemente è stato pubblicato uno studio focalizzato sulla qualità del risveglio post procedurale, valutata da parte del caregiver \cite{Cortellazzo2022}. Questo lavoro conferma l'evidenza di un elevato livello di soddisfazione parentale associato alla sedazione procedurale, indipendentemente dagli agenti farmacologici utilizzati, rilevato da alcuni studi precedenti, i quali tuttavia non hanno preso in considerazione  %In questo articolo sono anche stati esposti i fattori più frequentemente associati ad un basso livello di gradimento, che tuttavia si è complessivamente dimostrato elevato per tutti i regimi farmacologici confrontati.  

\begin{description}
\item[Metodi] 

%\subsubsection*{Metodi} 
Si tratta di uno studio osservazionale prospettico, condotto presso l'ospedale universitario materno infantile IRCSS Burlo Garofolo di Trieste, in cui sono stati confrontati quattro regimi farmacologici ---propofol, propofol più midazolam (\texttt{EV} o \texttt{OS}), ketamina più propofol, dexmedetomidina (\texttt{IN}) più midazolam (\texttt{OS})---\footnote{Legenda: \texttt{OS} = per via orale, \texttt{EV} = per via endovenosa, \texttt{IN} = per via intranasale.} utilizzati per la sedazione procedurale di bambini tra gli 0 e i 18 anni. La scelta del farmaco si è basata su diverse variabili, quali l'età del paziente, il livello di sedazione desiderato ed il tipo di procedura, oltre alla necessità di analgesia ed alla storia di eventi avversi in precedenti sedazioni. 
\\Allo scopo dello studio è stato sviluppato un questionario non validato\footnote{Al momento in letteratura non esistono linee guida per misurare il livello di soddisfazione in merito alla sedazione pediatrica.}, in cui è stato chiesto ai genitori dei piccoli pazienti arruolati nel lavoro di ricerca (in totale 655) di valutare il grado di soddisfazione della sedazione procedurale con un punteggio \texttt{NRS}\footnote{\emph{Numerical Rating Scale}, da 0 a 10.}, in due fasi: nel periodo intercorso dalla fine della procedura al completo risveglio del bambino ed il giorno dopo la procedura. In questo modo è stato possibile indagare sia la presenza di effetti avversi e la percezione del risveglio dal punto di vista del caregiver, sia la presenza di sintomi tardivi ed il grado di soddisfazione complessivo dell'esperienza di sedazione.

\item[Risultati] 
%\subsubsection*{Risultati}
Il grado di soddisfazione percepito dal caregiver durante la prima fase è stato molto elevato: ben il 77.7$\%$ degli intervistati ha dato una valutazione \texttt{NRS} $\geq$ 8, senza differenze significative tra il gruppo di soggetti sottoposti a procedure dolorose e quello in cui l'analgesia non era necessaria. In particolare, un livello \texttt{NRS} $\geq$ 8 è stato riscontrato nel 81.2$\%$ dei pazienti che hanno ricevuto solo propofol, nel 80.8$\%$ di coloro che hanno ricevuto propofol + midazolam, nel 79.1$\%$ del gruppo ricevente ketamina + propofol, mentre i livelli di soddisfazione più bassi sono stati osservati con la combinazione di dexmedetomidina + midazolam (66.1$\%$). Tuttavia, stratificando per l'età, in particolar modo restringendo l'analisi alla fascia d'età 0-4 anni, le differenze non risultano statisticamente significative. Lo stesso quadro viene confermato per la seconda fase, in cui viene valutata complessivamente l'esperienza: è stata infatti giudicata con un livello di gradimento altrettanto elevato: l'82.3$\%$ dei caregiver ha dato un punteggio \texttt{NRS} $\geq$ 8. Anche in questo caso l'associazione dexmedetomidina + midazolam ha mostrato gradi di soddisfazione più bassi (68.5$\%$) rispetto agli altri agenti farmacologici.
\\Gli effetti avversi più comunemente riscontrati dopo il risveglio sono: sonnolenza (55.4$\%$) ed irritabilità (24.6$\%$), seguiti meno frequentemente da irrequietezza ed agitazione, instabilità e vertigini, cefalea, alterazioni nell'appetito, nausea o vomito e molto raramente allucinazioni e distress respiratorio. Ad eccezione delle allucinazioni, non rilevate come sintomo tardivo, il giorno dopo la sedazione sono state riportate da alcuni genitori le precedentemente elencate reazioni avverse nello stesso ordine di frequenza, pur con un'incidenza molto più bassa. In questa ricerca l'insorgenza di effetti collaterali ha influenzato negativamente il grado di soddisfazione parentale ed è stata più frequentemente associata ai bambini che hanno ricevuto combinazioni di farmaci piuttosto che a coloro a cui è stato somministrato solo il propofol. Anche il tempo di recupero e di ripresa delle normali attività si è dimostrato essere più rapido (circa 1 ora) nei soggetti che hanno ricevuto solo il propofol, mentre è stato più lento (circa 3 ore) nei soggetti riceventi dexmedetomidina + midazolam. 

\newpage
\item[Discussione] 
%\subsubsection*{Discussione}
Questo studio ha dimostrato un elevato livello di soddisfazione parentale relativo alla sedazione pediatrica, indipendentemente dal regime farmacologico utilizzato e dalla presenza o meno di dolore associato alla procedura. I principali fattori che hanno influenzato negativamente il giudizio genitoriale sono legati alla comparsa di eventi avversi ed all'età del bambino: infatti i caregiver dei pazienti più piccoli hanno valutato l'esperienza di sedazione con punteggi più bassi.
Alla luce di ciò si possono giustificare i valori di gradimento inferiori associati alla somministrazione di dexmedetomidina + midazolam, combinazione frequentemente utilizzata nei bambini più piccoli, sottoposti a procedure di maggior durata (ad esempio RM). Oltretutto, è stata associata a tempi di risveglio più lunghi e ad una maggior incidenza di episodi di irritabilità. 

\item[Limiti]
%\subsubsection*{Limiti}
I principali limiti di questo studio riguardano la comprensibile inclinazione del genitore ad includere nel metro di giudizio diversi fattori, non strettamente legati alle caratteristiche del risveglio post sedazione e a valutare l'esperienza ospedaliera nel complesso. Ad esempio, alcuni elementi oggetto di bias possono essere la gravità della patologia del figlio, il tipo e la durata della procedura, precedenti esperienze di sedazione, il background culturale e l'interazione con il personale. 
\\Inoltre, poiché la scelta del farmaco è stata effettuata caso per caso, in base all'età ed al tipo di procedura, un altro limite di questa ricerca consiste nell'aver comparato gruppi farmacologici non omogenei in termini di numerosità ed età dei pazienti.
\item[Prospettive] 
%\subsubsection{Prospettive}

In conclusione, gli autori dello studio suggeriscono l'utilità di ulteriori indagini e ricerche al fine di confermare questi risultati e per discernere le componenti soggettive ed oggettive correlate alla valutazione della qualità del risveglio post procedurale. Questo lavoro di tesi si propone di rispondere, almeno parzialmente, a tale esigenza.


\end{description}

\end{comment}

\section{Scelta dell'infermiere come indicatore di qualità}

La scelta di valutare, in questo lavoro di tesi, la percezione del personale infermieristico deriva dalla necessità di ampliare ed oggettivare i risultati emersi negli studi riguardanti la soddisfazione parentale \cite{Cortellazzo2022, Lew2010, Connor2014}. Inoltre, esistono pochi dati in letteratura sull'opinione degli infermieri sulla pratica della sedazione pediatrica.\\
Mentre il caregiver viene facilmente influenzato da diversi fattori associati alla totalità del vissuto ospedaliero, dal servizio di parcheggio, alla simpatia del personale, finanche alla comparsa di complicanze correlate specificamente alla procedura; l'infermiere, in quanto figura professionale che assiste i pazienti durante tutte le fasi della procedura: dall'ammissione, al posizionamento dell'accesso venoso, al monitoraggio dei parametri vitali e alla somministrazione dei farmaci prima e durante la sedazione, fino poi al completo risveglio ed alla dimissione, risulta il candidato più adatto a giudicare in maniera oggettiva gli elementi associati alla sedazione ed al risveglio post procedurale. Inoltre, mentre il genitore esprime il proprio parere basandosi su un'unica o su limitate esperienze di analgosedazione, l'infermiere partecipa a molteplici sedazioni procedurali ogni mese ed è quindi in grado di confrontare diversi regimi farmacologici ed esprimere un giudizio complessivo basato sull'efficacia della sedazione, sulla facilità e sulla sicurezza della via di somministrazione, oltre che sull'incidenza media e la severità degli effetti avversi associati.
\\In conclusione, l'opinione dell'infermiere riveste un ruolo chiave non solo al fine di ottenere delle indicazioni utili per la scelta farmacologica ma anche come occasione per integrare le conoscenze mediche alle osservazioni di una figura di riferimento fondamentale per il paziente e la famiglia, quale quella dell'infermiere. La collaborazione tra la professionalità medica ed infermieristica rappresenta il presupposto migliore per garantire alla comunità una sempre maggiore attenzione alla qualità di cura. 

\section{Standard di cura per la sedazione all'Istituto Materno Infantile Burlo Garofolo}

La sedazione procedurale pediatrica per indagini diagnostiche o interventi terapeutici, più o meno dolorosi, viene quotidianamente svolta in questo centro universitario, sia da pediatri che da specializzandi pediatrici. \`E stata, infatti, istituita un'Unità di Sedazione Pediatrica che segue tutte le procedure svolte al di fuori della sala operatoria, che richiedano analgesia e/o ansiolisi fino alla sedazione profonda\footnote{I diversi livelli della sedazione e le varie fasi del processo sedativo: dalla valutazione preprocedurale, al monitoraggio strumentale ed obiettivo fino alla dimissione del bambino, oltre che i possibili effetti avversi correlati, sono ampiamente descritti nell'appendice A.}. Quest'Unità è supervisionata da due anestesisti pediatrici, che garantiscono la formazione degli specializzandi, il monitoraggio durante le procedure e il pronto intervento in caso d'emergenza. \`E, inoltre, assistita da infermieri pediatrici con una preparazione specifica nell'ambito delle sedazioni procedurali. Quest'ultimi esercitano un ruolo fondamentale in tutte le fasi della procedura: accolgono le famiglie e i bambini, rilevano peso, altezza e i parametri vitali dei piccoli pazienti, inoltre posizionano l'accesso venoso periferico, applicano le tecniche di distrazione e partecipano alla somministrazione dei farmaci e al monitoraggio dei pazienti durante e dopo la procedura. Di conseguenza, gli infermieri possiedono una panoramica completa sull'esperienza sedativa, dal ricovero alla dimissione dei pazienti, e il loro feedback è di particolare rilevanza per i medici sedatori. 
\\I principali agenti analgesici e sedativi utilizzati, da soli o in combinazione, sono il propofol, la ketamina, il midazolam e la dexmedetomidina: le loro caratteristiche farmacocinetiche e farmacodinamiche, le indicazioni, la posologia e gli effetti avversi sono riportati nell'appendice B. La scelta farmacologica verte su innumerevoli fattori, tra cui l'età del paziente, la storia clinica comprensiva di precedenti esperienze di sedazione ed eventuali reazioni avverse, la presenza di patologie in atto o di anomalie anatomiche che potrebbero rendere difficile la manovra di intubazione in caso di necessità: tutti questi elementi sono raccolti prima della procedura per mezzo di un'attenta anamnesi ed un accurato esame obiettivo. Oltre a ciò, risultano altrettanto importanti per la selezione del regime sedativo più adatto: il tipo di procedura, il livello di dolore atteso e la difficoltà prevista nel posizionamento del catetere venoso periferico, procedura che può risultare impegnativa, ad esempio, nei bambini con DIVA score elevato\footnote{\emph{Difficult Intravenous Access score}: si tratta di un sistema di punteggio basato su criteri clinici facili da applicare, che permettono di predire la difficoltà di inserzione del catetere venoso periferico per ciascun paziente pediatrico \cite{Yen2008}.} o con una storia di precedenti tentativi eseguiti con insuccesso. In questi casi si possono scegliere uno o più agenti farmacologici in combinazione, che consentano la somministrazione per vie alternative a quella endovenosa. 
Generalmente il propofol viene scelto per procedure non dolorose o solo minimamente dolorose, in cui è importante che il paziente non sia facilmente risvegliabile, quali endoscopie, biopsie o aspirati midollari e rachicentesi. Può essere inoltre associato alla ketamina per lo svolgimento di procedure in cui è richiesto un grado di analgesia maggiore, quali biopsie epatiche, renali od ossee, riduzioni di fratture ed estrazioni dentarie. In altri casi, invece, risulta preferibile l'utilizzo della sola ketamina al fine di evitare artefatti in corso di esami neurofisiologici, quali l'elettromiografia o i potenziali evocati. La dexmedetomidina, invece, determinando uno stato sedativo simile al sonno fisiologico, viene usata principalmente nei bambini piccoli o non collaboranti per gli studi radiologici di lunga durata, quali risonanza magnetica o scintigrafia, dove non si verificano stimoli tattili o dolorifici che potrebbero risvegliare il paziente. Inoltre, sia la dexmedetomidina sia il midazolam possono essere utilizzati per via intranasale o per via orale (solo midazolam) per facilitare il posizionamento dell'accesso venoso periferico nei pazienti più piccoli (sotto i sei anni d'età) o con DIVA score elevato. In aggiunta il midazolam può essere somministrato come premedicazione per via endovenosa anche nei bambini più grandi che sperimentano elevati livelli d'ansia prima della procedura, al fine di rendere meno traumatica l'esperienza complessiva. Tutte le punture vengono effettuate su zone di cute dove è stata precedentemente applicata un'emulsione anestetica, la più comunemente utilizzata è la crema EMLA: una miscela eutettica di lidocaina e prilocaina. Infine, vengono routinariamente adoperate come strumento complementare diverse tecniche di distrazione attive e passive, adattate in base all'età del paziente. Tale approccio non farmacologico comprende ad esempio l'impiego del gioco, la visione di video o cartoni, la lettura di un libro, l'ascolto di musica o più semplicemente l'instaurazione di un dialogo in cui viene deviata l'attenzione del bambino tramite varie domande personali o d'immaginazione. Inoltre, viene richiesto un contributo attivo anche ai genitori dei piccoli pazienti sia durante l'inserimento della cannula venosa sia durante la somministrazione dei farmaci. Al fine di ridurre l'agitazione preprocedurale e offrire un'esperienza quanto più serena possibile è, infatti, consentito loro di rimanere al fianco del figlio fino al raggiungimento di un adeguato livello di sedazione. 

\section{Scopi della tesi}

La presente analisi si immerge nell'ambito delle sedazioni procedurali pediatriche, campo in evoluzione e crescente diffusione in tutto il mondo ma i cui fattori di qualità risultano tutt'ora poco esplorati dalla comunità scientifica.
%dal punto di vista della qualità dell'esperienza di sedazione.
Per tale ragione questo lavoro mira a valutare il livello di soddisfazione del personale infermieristico associato ai diversi regimi farmacologici utilizzati durante le sedazioni effettuate fuori dalla sala operatoria. Nello specifico verranno indagati la percezione del profilo di sicurezza dei farmaci, le preferenze relative al sito di somministrazione, la frequenza d'insorgenza di eventuali effetti avversi e il giudizio relativo alla sedazione nel suo complesso.
L'obiettivo finale è quello di fornire delle indicazioni complete ed oggettive ai sedatori, affinché le possano applicare, laddove possibile, nella scelta farmacologica, con lo scopo di offrire al paziente ed alla famiglia uno standard di cura ed una qualità del servizio sanitario sempre maggiori.

%Inoltre, i risultati di questo lavoro verranno confrontati con l'esigua letteratura nota, in particolare con lo studio sopra descritto \cite{Cortellazzo2022}, in cui è stata scelta la figura del caregiver come indicatore di qualità della sedazione. 
%I principali obiettivi di quest'indagine concernono la volontà di fornire indicazioni quanto più oggettive in merito alla qualità della sedazione procedurale e del risveglio successivo. Mira, quindi, a mettere a confronto i quattro farmaci sedativi, analgesici e dissociativi più ampiamente utilizzati, avvalendosi della percezione infermieristica come indicatore di soddisfazione. 
%tuttavia rimane ancora inesplorato dalla comunità scientifica bensì, tuttavia, riveste un ruolo di rilevo sia  considerata la sempre maggiore diffusione e pratica in tutto il mondo della sedazione procedurale sia tenuta presente la volontà, in un'ottica di progresso, 
%Quest'indagine si immerge in un campo ancora inesplorato dalla comunità scientifica bensì, tuttavia, riveste un ruolo di rilevo sia  considerata la sempre maggiore diffusione e pratica in tutto il mondo della sedazione procedurale sia tenuta presente la volontà, in un'ottica di progresso, di migliorare le prestazioni offerte: 

%\subsection{A subsection}

%\lipsum[3]

%\subsubsection{A subsubsection}

%\lipsum[4]