\chapter{Conclusioni}

Questo studio ha mostrato che, tra i farmaci più comunemente utilizzati per la sedazione pediatrica al di fuori della sala operatoria, il propofol, il midazolam e la dexmedetomidina sono i più apprezzati dagli infermieri. Diversamente, la ketamina è stata associata a livelli di soddisfazione significativamente inferiori. Ad ogni modo, la sicurezza percepita durante la sedazione dagli intervistati è risultata elevata per tutti i regimi farmacologici testati. Ad incidere maggiormente sul giudizio degli infermieri sono stati i principali effetti collaterali riscontrati durante e dopo la procedura e la qualità generale del risveglio. 
In ultima analisi, questo studio ha messo in luce la rilevanza dell'opinione infermieristica nell'ambito delle sedazioni procedurali pediatriche, con l'auspicio che una prospettiva multi-professionale possa, in generale, essere presa in considerazione anche in altri campi, al fine di favorire la cooperazione tra professionisti e migliorare la qualità del servizio offerto. 