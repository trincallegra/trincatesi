\chapter{Conclusioni}

Questo studio ha mostrato che il propofol, il midazolam e la dexmedetomidina sono i farmaci, utilizzati per la sedazione pediatrica al di fuori della sala operatoria, più apprezzati dagli infermieri. Diversamente, la ketamina è stata correlata a livelli di soddisfazione significativamente inferiori. Ad influenzare questa risposta sono stati principalmente gli effetti collaterali e la qualità del risveglio. Ad ogni modo, il livello di sicurezza percepito dal personale infermieristico è risultato elevato, indipendentemente dal regime farmacologico utilizzato. Nonostante i regimi farmacologici cambieranno con il passare degli anni, parallelamente all'imprescindibile evoluzione scientifica, la comprensione delle componenti della qualità di cura permarrà e rappresenterà un punto di partenza per la ricerca futura. Nello specifico, questo studio ha dimostrato la rilevanza dell'opinione infermieristica nell'ambito delle sedazioni procedurali pediatriche, con l'auspicio che la prospettiva infermieristica possa essere presa in considerazione anche in altri campi, al fine di favorire la cooperazione tra professionisti e migliorare la qualità del servizio offerto. 