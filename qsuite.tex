%%%%%%%%%%%%%%%%%%%%%%%%%%%%%%%%%
%% Required package inclusions %%
%%%%%%%%%%%%%%%%%%%%%%%%%%%%%%%%%

\usepackage{wasysym} % provides $\ocircle$ and $\Box$
\usepackage{forloop} % used for \Qrating and \Qlines
\usepackage[breakable]{tcolorbox} % xcolor required 

%%%%%%%%%%%%%%%%%%%%%%%%%%%%%%%%%%%%%%%%
%% Special colorful boxed environment %%
%%%%%%%%%%%%%%%%%%%%%%%%%%%%%%%%%%%%%%%%
\newenvironment{survey}[2][QUESTIONARIO] 
{
\newcommand{\Qcolor}{#2}
\begin{tcolorbox}
[breakable,
title=#1,
colback=\Qcolor!25,
colframe=\Qcolor!25,
coltitle=black,
colbacktitle=\Qcolor!50,
fonttitle=\sffamily\mdseries\scshape,
halign title=flush center,
arc=3mm,
boxrule=0pt]
\begin{enumerate}
[leftmargin=1.5em,
itemsep=1.5em]
\newcommand{\Query}{\item} % fancy alias
}
{
\end{enumerate}
\end{tcolorbox}
} % USAGE: \begin{survey}[title]{colorname}

%%%%%%%%%%%%%%%%%%%%%%%%%%%%%%%%%%%%%%%%%%%%%%%%%%%%
%% Beginning of questionnaire command definitions %%
%%%%%%%%%%%%%%%%%%%%%%%%%%%%%%%%%%%%%%%%%%%%%%%%%%%%

%% \Qbox -> Empty box to be ticked.
%% > Used both by direct call and by \Qlist itemization.
\fboxsep=1mm \fboxrule=0.6pt
\newcommand{\Qbox}
{\fcolorbox{black}{\Qcolor!8}{\color{\Qcolor!8}$\ocircle$}}
%% > or replace $\ocircle$ with $\Box$, making the resulting box a little wider...

%% \Qlist -> This is an environment very similar to itemize but with
%% \Qbox in front of each list item. Useful for classical multiple
%% choice. Change leftmargin and topsep according to your taste.
\newenvironment{Qlist}{%
\renewcommand{\labelitemi}{\Qbox}
\begin{itemize}[leftmargin=2em]   %[leftmargin=2.57em]%,topsep=-.5em]
}{%                               ^matches itemize    ^no good
\end{itemize}
}

%% \Qnum -> Similar to \Qbox, but it contains a number.
%% > Used within \Qrating.
\newcommand{\Qnum}[1]{\colorbox{\Qcolor!8}{#1}}

%% \Qrating{N} -> Automatically create a rating scale with N steps.
%% > like this: [1] [2] [3] … [N] .
\newcounter{qr}
\newcommand{\Qrating}[1]{\forloop{qr}{0}{\value{qr}<#1}{~\Qnum{\arabic{qr}}}}

%% \Qline -> Again, this is very simple. It helps setting the line thickness globally.
%% > You can change the default width (=\linewidth) with the \Qline{WIDTH} syntax
%% > Used both by direct call and \Qlines newline-iterator.
\newcommand{\Qline}[1]{\noindent\rule{#1}{0.6pt}}

%% \Qlines{N} -> Insert N lines with width=\linewidth (default \Qline). 
%% > You can change the \vskip value to tweak the vertical interspacing.
\newcounter{ql}
\newcommand{\Qlines}[1]{\forloop{ql}{0}{\value{ql}<#1}{\vskip0em\Qline{\linewidth}}}


%% \Qtab -> A "tabulator". The first argument is the distance from the left.
%% The second argument is the content to be indented [no line skip].
\newlength{\qt}
\newcommand{\Qtab}[2]{
\setlength{\qt}{\linewidth}
\addtolength{\qt}{-#1}
\hfill\parbox[t]{\qt}{\raggedright #2}
}

%%%%%%%%%%%%%%%%%%%%%%%%%%%%%%%%%%%%%%%%%%%%%%
%% End of questionnaire command definitions %%
%%%%%%%%%%%%%%%%%%%%%%%%%%%%%%%%%%%%%%%%%%%%%%
%%
%% •••••••••••••••••••••••••••••••••••••••••••
%% COPYRIGHT
%% ···········································
%% © 2010-2012, Sven Hartenstein [original]
%% mail@svenhartenstein.de
%% http://www.svenhartenstein.de
%% ···········································
%% © 2022, G. Bellomia [heavy customization]
%% gabriele.bellomia@sissa.it
%% https://github.com/bellomia
%% •••••••••••••••••••••••••••••••••••••••••••
%%
%% Please be warned that this is NOT a full-featured framework for
%% creating (all sorts of) questionnaires. Rather, it is a small
%% collection of LaTeX commands that I found useful when creating a
%% questionnaire. Feel free to copy and adjust any parts you like.
%% Most probably, you will want to change the commands, so that they
%% fit your taste.
%%%%%%%%%%%%%%%%%%%%%%%%%%%%%%%%%%%%%%%%%%%%%%%%%%%%%%%%%%%%%%%%%%LICENSE